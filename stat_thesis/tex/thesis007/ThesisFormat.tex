%	Written by 유재성 (2014. 11. 04)
% 이 양식을 사용하기 위해 알아야할 것

% 1. TeX 컴파일러는 ko.TeX Live가 아니라 MikTeX을 사용해야함

% 2. 한글을 사용하기 위해서 kotex 패키지를 use하는데
% 			Miktex 2.9부터는 kotex에 버그가 있어서 잘 안될 것임.
%			그럴 때는 아래 링크대로 kotex을 따로 살치할 것.
% 			http://doeun.blogspot.kr/2012/08/miktex-29-kotex.html

% 3. 한글 논문을 위한 부분, 영문 논문을 위한 부분이 따로 있으니
%			알아서 필요한 부분만 추려 쓸 것.
%			다만 공통되는 부분은 따로 추릴 필요는 없을 것임.

\documentclass[12pt,a4paper,oneside]{book}

% \usepackage[UTF8]{ctex}
% \usepackage[UTF8]{ctexart}

\usepackage[pdftex]{graphicx}
\usepackage{epstopdf}
\usepackage{amsthm}
\usepackage{amsmath}

\usepackage{geometry} % Package re-sizing page for b5paper
\usepackage[hangul]{kotex} % Korean support
\usepackage{natbib} % cite style
\usepackage{paralist} % package for inline enumerate, i) ii)...
\usepackage{graphicx} % Allows including images
\usepackage{amssymb} % for math symbols

\usepackage{tablefootnote} % to use table foot note
\pagestyle{plain}



% MATH  ------------------------------------------------------------------

\newcommand{\spart}[1]{\left(#1\right)}
\newcommand{\ppart}[1]{\left[#1\right]}
\newcommand{\norm}[1]{\left|#1\right|}
\newcommand{\med}{\mathrm{median}}
\newcommand {\bm}[1]{\mbox{\boldmath{$#1$}}}
\newcommand {\bb} {{\bm{\beta}}}
\newcommand {\argmin} {\mathop{\rm{argmin}}}

% Line spacing -----------------------------------------------------------
\newlength{\defbaselineskip}
\setlength{\defbaselineskip}{\baselineskip}
\newcommand{\setlinespacing}[1]%
           {\setlength{\baselineskip}{#1 \defbaselineskip}}
\newcommand{\doublespacing}{\setlength{\baselineskip}%
                           {2.0 \defbaselineskip}}
\newcommand{\singlespacing}{\setlength{\baselineskip}{\defbaselineskip}}

%\setlength{\tclineskip}{1.66\baselineskip}
\linespread{1.6}

% Graphic  -----------------------------------------------------------
\DeclareGraphicsRule{.png}{bmp}{}{}

% Bibliography ------------------------------------------------------------
\renewcommand\bibname{References}

% Table of Contents

% Theorem Style
% THEOREMS ---------------------------------------------------------------
\theoremstyle{plain}
\newtheorem{thm}{Theorem}
\newtheorem{cor}[thm]{Corollary}
\newtheorem{lem}[thm]{Lemma}
\newtheorem{prop}[thm]{Proposition}
%
\theoremstyle{definition}
\newtheorem{defn}{Definition}[chapter]
%
\theoremstyle{remark}
\newtheorem{rem}{Remark}[chapter]
%
\theoremstyle{definition}
\newtheorem{exam}{Example}[chapter]
%
\numberwithin{equation}{chapter}
%\renewcommand{\theequation}{\thesection.\arabic{equation}}
% \setlength{\textwidth 140mm} \setlength{\textheight 200mm}




\begin{document}
%\iffalse
% Cover





%%%%% Cover 1
 \linespread{1.0}
 \thispagestyle{empty}
 \begin{center}
 {\Large Thesis for the Degree of Master}
 \end{center}
 \vspace{20mm}
 \begin{center}
 \LARGE A Study on Comparison of\\ Bayesian Network Structure Learning Algorithm\\ for Selecting Appropriate Model
 \end{center}
 \vspace{20mm}
  \begin{center}
 {\Large by}
 \end{center}
 \begin{center}
 {\Large YOO, JAE SEONG}
 \end{center}
 \vspace{40mm}
 \begin{center}
 {\Large Department of Statistics}
 \end{center}
 \begin{center}
 {\Large Graduate School}
 \end{center}
 \begin{center}
 {\Large Korea University}
 \end{center}
  \vspace{5mm}
 \begin{center}
 {\Large December, 2014}
 \end{center}
 \linespread{1.6}





\newpage{}
%%%%% Cover 1
 \linespread{1.0}
 \thispagestyle{empty}
 \vspace{30mm}
 \begin{center}
 {\Large 碩 士 學 位 論 文}
 \end{center}
 \vspace{27mm}
 \begin{center}
 \LARGE A Study on Comparison of\\ Bayesian Network Structure Learning Algorithm\\ for Selecting Appropriate Model
 \end{center}
 \vspace{80mm}
 \begin{center}
 {\Large 高麗大學校 大學院}
 \end{center}
 \begin{center}
 {\Large 統 計 學 科}
 \end{center}
 \begin{center}
 {\Large 兪 \quad\quad 在 \quad\quad 成}
 \end{center}
  \vspace{7mm}
 \begin{center}
 {\Large 2014年 \quad 12月 \quad 日}
 \end{center}
 \linespread{1.6}





\newpage{}
%%%%% Cover 2
 \linespread{1.0}
 \thispagestyle{empty}
 \vspace{15mm}
 \begin{center}
 {\Large 崔 太 連 敎授指導 \\ 碩 士 學 位 論 文
}
 \end{center}
 \vspace{20mm}
 \begin{center}
  \LARGE A Study on Comparison of\\ Bayesian Network Structure Learning Algorithm\\ for Selecting Appropriate Model
 \end{center}
 \vspace{20mm}
 \begin{center}
 {\Large 이 論文을 統計學碩士 學位論文으로 提出함}
 \end{center}
 \vspace{15mm}
 \begin{center}
 {\Large 2014年 \quad 12月 \quad 日}
 \end{center}
 \vspace{20mm}
 \begin{center}
 {\Large 高麗大學校 大學院}
 \end{center}
 \begin{center}
  {\Large 統 計 學 科}
 \end{center}
 \begin{center}
 {\Large 兪 \quad\quad 在 \quad\quad 成}
 \end{center}
 \linespread{1.6}





\newpage{}		
%%%%% Cover 3
 \linespread{1.0}
 \thispagestyle{empty}
 \vspace{20mm}
 \begin{center}
 {\LARGE 兪在成의 統計學碩士 學位論文 \\ 審査를 完了함}
 \end{center}
 \vspace{50mm}
 \begin{center}
{\Large 2014年 \quad 12月 \quad 日}
 \end{center}
 \vspace{55mm}
 \begin{center}
 {\Large  \underline{委員長 \hspace{60mm} (印)} \\ \vspace{7mm}
          \underline{委 \quad 員\hspace{60mm} (印)} \\ \vspace{7mm}
          \underline{委 \quad 員\hspace{60mm} (印)} \\ }
 \end{center}
 \linespread{1.6}





\newpage{}
%%%%% Abstract
    \pagenumbering{roman}
    \typeout{Abstract}

\begin{center}
{\Large 국 문 초 록}
\end{center}
\vspace{1cm} {\small \ \indent 본 논문은 부분선형모형을 가우스 확률과정(Gaussian process)을 이용하여 베이지안 접근방식을 고려하였다. 특히 부분선형모형 중 비모수 함수 항을 여러 개의 1차원 함수의 가법 형태인 부분선형가법모형으로 구체화하여 구현하였다. 또한, 오차분포의 가정을 비모수적 접근을 시도하기 위해 디리슈레 과정 혼합(Dirichlet process mixture, DPM)을 사용한 부분선형가법모형도 시도해 보았다. 모의 실험 및 사례연구에서는 기존에 알려져있는 부분선형가법모형을 구현하는 서로 다른 방법들을 비교하였고 가우스 확률과정을 이용한 부분선형가법모형의 성능과 한계점을 모색해 보았다.
}

 %\include{ABSTRACT_ENG}
%\addtocounter{page}{-1}
 \tableofcontents
 \newpage
 \listoftables
 \newpage
 \listoffigures
%\fi

\newpage

\pagenumbering{arabic} \setcounter{page}{1}
%%%%%%%%%%%%%%%%%%%%%%%%%%%%%%%%%%
% 여기부터 본문입니다.
%%%%%%%%%%%%%%%%%%%%%%%%%%%%%%%%%%

\chapter{끎말}

\section{베이지안 네트워크}

{}\

부분 선형 가법 모형(partially linear additive model)은 모수적 회귀 모형과 비모수적 회귀 모형이 결합된 준모수 회귀 모형인 부분 선형 모형(partially linear model)을 확장한 형태로 하나 이상의 비모수적 회귀모형 부분을 가법적 결합형태로 일반화시킨 모형이다. 부분 선형 가법 모형은 부분선형 모형과 마찬가지로 반응변수를 설명하는 두가지 형태의 공변량(covariate)에 대해서 하나의 공변량은 반응변수와 선형관계, 다른 하나의 공변량은 반응변수와 비모수적인 함수 형태로 연결되는 함수적 관계를 설명한다.
%%%%%%%%%%%%%%%%%%%%%%%%%%%%%%%%%%
\newpage % for correct header (if you realy have empty page, delete this one.)



\chapter*{References}
\begin{bibliography}{99}
\addcontentsline{toc}{chapter}{References}
\begin{enumerate}

    \bibitem{ref1} 저자 (0000). 제목, \emph{저널}, Vol. 18, No. 3, 543-553.

\end{enumerate}
\end{bibliography}


\end{document}
